\documentclass[11pt]{article}
\usepackage[margin=1in]{geometry}
\usepackage{amsmath, amssymb}
\usepackage{booktabs}
\usepackage{hyperref}

\title{Lifecycle Investment Model Specification\\
\large Random Walk Interest Rate Case ($\phi = 1$)}
\author{FINC450}
\date{}

\begin{document}

\maketitle

\section{Data Generating Process}

This document describes the lifecycle investment model with \textbf{random walk interest rates} ($\phi = 1$). Under this specification, rates follow a pure random walk with no mean reversion.

\textbf{Notation.} The model uses continuous compounding throughout. Zero-coupon bond prices, present values, and durations are computed using exponential discounting $e^{-\tau r}$ rather than discrete discounting $(1+r)^{-\tau}$.

\subsection{Interest Rate Process (Random Walk)}

The short-term interest rate evolves as:
\begin{equation}
r_{t+1} = r_t + \sigma_r \varepsilon^r_{t+1}, \quad \varepsilon^r_{t+1} \sim N(0,1)
\end{equation}
where:
\begin{itemize}
    \item $\sigma_r$ = interest rate volatility
    \item No drift term (expected change is zero)
    \item No mean reversion ($\phi = 1$ implies persistence = 1)
    \item Rates are unconstrained (no floor or ceiling)
\end{itemize}

\textbf{General AR(1) case.} The code supports mean-reverting rates $r_{t+1} = \bar{r} + \phi(r_t - \bar{r}) + \sigma_r \varepsilon^r_{t+1}$ with $0 < \phi < 1$, but the default is $\phi = 1$.

\textbf{Key implication:} With $\phi = 1$, the effective duration of a $\tau$-year zero-coupon bond equals $\tau$ (maturity), and the zero-coupon price simplifies to $P(\tau) = e^{-\tau r}$.

\subsection{Stock Returns}

Stock returns from $t$ to $t+1$:
\begin{equation}
R^s_{t+1} = r_t + \mu_s + \sigma_s \varepsilon^s_{t+1}
\end{equation}
where:
\begin{itemize}
    \item $\mu_s$ = equity risk premium (excess return over short rate)
    \item $\sigma_s$ = stock volatility
    \item $\text{Corr}(\varepsilon^r_{t+1}, \varepsilon^s_{t+1}) = \rho$ (default 0)
\end{itemize}

\subsection{Bond Returns (Duration Approximation)}

For a bond portfolio with duration $D$, the return from $t$ to $t+1$ is:
\begin{equation}
R^b_{t+1} \approx r_t + \mu_b - D \cdot \Delta r_{t+1}
\end{equation}
where:
\begin{itemize}
    \item $D$ = duration of the bond portfolio (a fixed parameter)
    \item $\mu_b$ = bond risk premium (term spread over short rate)
    \item $\Delta r_{t+1} = r_{t+1} - r_t = \sigma_r \varepsilon^r_{t+1}$
\end{itemize}

Substituting the rate change:
\begin{equation}
R^b_{t+1} \approx r_t + \mu_b - D \cdot \sigma_r \varepsilon^r_{t+1}
\end{equation}

\textbf{Intuition:} When rates rise ($\varepsilon^r_{t+1} > 0$), bond prices fall, generating a negative capital gain proportional to duration.

\textbf{Bond risk premium.} The bond excess return is parameterized via the Bond Sharpe ratio $\text{SR}_b$:
\begin{equation}
\mu_b = \text{SR}_b \cdot D \cdot \sigma_r
\end{equation}
The bond return volatility is $\sigma_b = D \cdot \sigma_r$, so $\mu_b = \text{SR}_b \cdot \sigma_b$. The default is $\text{SR}_b = 0$ (expectations hypothesis: no term premium).

%----------------------------------------------------------------------
\section{Earnings and Expense Profiles}
%----------------------------------------------------------------------

\subsection{Earnings Profile}

Earnings follow a hump-shaped profile over the working life ($t_0 \leq t < T_R$):
\begin{equation}
Y_t = \begin{cases}
Y_0 (1 + g_Y)^{t - t_0} & \text{if } t \leq t_{\text{peak}} \\
Y_{\text{peak}} (1 - d_Y)^{t - t_{\text{peak}}} & \text{if } t > t_{\text{peak}}
\end{cases}
\end{equation}
where:
\begin{itemize}
    \item $Y_0$ = initial annual earnings
    \item $g_Y$ = real earnings growth rate (default 0, flat)
    \item $t_{\text{peak}}$ = age at peak earnings (default = $T_R$, so earnings are flat)
    \item $d_Y$ = decline rate after peak (default 0)
\end{itemize}

After retirement ($t \geq T_R$): $Y_t = 0$.

\subsection{Risky Human Capital (Wage Shocks)}
\label{sec:wage_shocks}

When $\beta_{HC} > 0$, earnings are subject to permanent shocks correlated with stock returns. The log wage level follows:
\begin{equation}
\ln W_t = \ln W_{t-1} + \beta_{HC} \cdot \sigma_s \cdot \varepsilon^s_{t-1}
\end{equation}
where $W_0 = 1$ (no initial shock). Actual earnings are:
\begin{equation}
\tilde{Y}_t = Y_t \cdot W_t = Y_t \cdot \exp\!\left(\sum_{k=0}^{t-1} \beta_{HC} \cdot \sigma_s \cdot \varepsilon^s_k\right)
\end{equation}

This captures the idea that human capital has systematic risk: when stock markets do well, wages tend to rise, and vice versa. The shock is permanent (a random walk in log wages), creating path dependence in lifetime earnings.

\subsection{Expense Profile}

Subsistence expenses are:
\begin{equation}
E_t = \begin{cases}
E_0 (1 + g_E)^{t - t_0} & \text{if } t < T_R \\
E_R & \text{if } t \geq T_R
\end{cases}
\end{equation}
where $E_0$ = base working expenses, $g_E$ = real expense growth (default 0), and $E_R$ = fixed retirement expenses.

%----------------------------------------------------------------------
\section{Present Values and Durations}
%----------------------------------------------------------------------

\subsection{Zero-Coupon Bond Pricing}

Under the term structure model with $\phi = 1$, the price of a $\tau$-year zero-coupon bond is:
\begin{equation}
P(\tau, r_t) = e^{-\tau r_t}
\end{equation}
This is standard continuous-time discounting at the current short rate.

\subsection{Human Capital}

Human Capital (HC) is the present value of future labor earnings, discounted at the CAPM-adjusted rate:
\begin{equation}
HC_t = \sum_{s=t}^{T_R-1} \tilde{Y}_s \cdot e^{-(s-t)(r_t + \beta_{HC} \mu_s)}
\end{equation}
where $\beta_{HC} \mu_s$ is the CAPM spread: human capital with stock-market exposure commands a higher discount rate, reducing its present value.

When $\beta_{HC} = 0$ (riskless HC), this simplifies to $HC_t = \sum_{s=t}^{T_R-1} Y_s \, e^{-(s-t) r_t}$.

After retirement ($t \geq T_R$): $HC_t = 0$.

\subsection{Present Value of Expenses (Liability)}

The present value of future subsistence expenses:
\begin{equation}
L_t = \sum_{s=t}^{T-1} E_s \cdot e^{-(s-t) r_t}
\end{equation}

\subsection{Duration of HC and Expenses}

Duration measures the sensitivity of present value to changes in the short rate. With $\phi = 1$, the duration of a cashflow stream is the present-value-weighted average maturity:
\begin{equation}
D^{HC}_t = \frac{\sum_{s=t}^{T_R-1} (s-t) \cdot \tilde{Y}_s \cdot e^{-(s-t)(r_t + \beta_{HC} \mu_s)}}{HC_t}
\end{equation}
\begin{equation}
D^{L}_t = \frac{\sum_{s=t}^{T-1} (s-t) \cdot E_s \cdot e^{-(s-t) r_t}}{L_t}
\end{equation}

\textbf{Duration cap.} An optional parameter $D_{\max}$ caps computed durations: $D \leftarrow \min(D, D_{\max})$. This prevents bond-equivalent fractions from exceeding 1 when HC or expense duration exceeds bond duration.

%----------------------------------------------------------------------
\section{Portfolio Allocation Rule}
%----------------------------------------------------------------------

\subsection{Net Worth}

Net worth is defined as assets minus liabilities:
\begin{equation}
NW_t = FW_t + HC_t - L_t
\end{equation}
where:
\begin{itemize}
    \item $FW_t$ = Financial Wealth (liquid portfolio)
    \item $HC_t$ = Human Capital (implicit asset)
    \item $L_t$ = Present value of future expenses (liability)
\end{itemize}

This is the amount available for discretionary consumption beyond subsistence.

\subsection{Target Total Allocation (Mean-Variance Optimization)}

The optimal portfolio weights from mean-variance optimization:
\begin{equation}
\mathbf{w}^* = \frac{1}{\gamma} \Sigma^{-1} \boldsymbol{\mu}
\end{equation}

\textbf{Explicit 2-asset system.} Cash earns the risk-free rate, so we solve for stock and bond weights over cash. The excess return vector is:
\begin{equation}
\boldsymbol{\mu} = \begin{pmatrix} \mu_s \\ \mu_b \end{pmatrix}
\end{equation}

The covariance matrix of risky asset returns uses $\sigma_b = D \cdot \sigma_r$ for bond volatility:
\begin{equation}
\Sigma = \begin{pmatrix}
\sigma_s^2 & -D \, \sigma_s \, \sigma_r \, \rho \\
-D \, \sigma_s \, \sigma_r \, \rho & (D \, \sigma_r)^2
\end{pmatrix}
\end{equation}

The off-diagonal term reflects $\text{Cov}(R^s, R^b) = -D \, \sigma_s \, \sigma_r \, \rho$: the negative sign arises because rising rates hurt bond returns.

Solving $\mathbf{w}^* = \frac{1}{\gamma} \Sigma^{-1} \boldsymbol{\mu}$ gives the optimal stock and bond weights $(w^*_s, w^*_b)$, with cash as the residual:
\begin{equation}
w^*_c = 1 - w^*_s - w^*_b
\end{equation}

These MV weights are used \emph{unconstrained} as target allocations. Constraints are applied only to the final dollar-level portfolio holdings (see Section~\ref{sec:weight_truncation}).

\subsection{Human Capital Decomposition}

HC has stock exposure $\beta_{HC}$ and interest-rate duration $D^{HC}_t$. The bond hedge uses the \emph{full} HC (not just the non-stock portion), because all of HC has interest-rate duration regardless of its stock beta. When $\beta_{HC} > 0$, the higher CAPM discount rate ($r + \beta_{HC}\mu_s$) naturally shortens $D^{HC}_t$ and reduces $HC_t$, so the bond hedge is already smaller:
\begin{align}
HC^{\text{stock}} &= \beta_{HC} \cdot HC_t \\
HC^{\text{bond}} &= HC_t \cdot \frac{D^{HC}_t}{D} \\
HC^{\text{cash}} &= HC_t - HC^{\text{stock}} - HC^{\text{bond}}
\end{align}
where $D$ = duration of bonds in the financial portfolio. The cash component is a residual and can be negative when the stock and bond components together exceed total HC (e.g., when $\beta_{HC}$ is large and $D^{HC}_t / D$ is close to 1).

\subsection{Expense Liability Decomposition}

The expense liability is decomposed by duration matching:
\begin{align}
L^{\text{bond}} &= L_t \cdot \frac{D^{L}_t}{D} \\
L^{\text{cash}} &= L_t \cdot \left(1 - \frac{D^{L}_t}{D}\right)
\end{align}

\subsection{Target Financial Holdings (Surplus Optimization)}

We define \textbf{surplus} as the positive part of net worth:
\begin{equation}
S_t = \max(0,\; NW_t) = \max(0,\; FW_t + HC_t - L_t)
\end{equation}

The target allocation is applied to surplus rather than raw net worth. This prevents the optimizer from taking leveraged bets when the agent is ``underwater'' ($NW < 0$).

\begin{align}
\text{target\_fin\_stock} &= w^*_s \cdot S_t - HC^{\text{stock}} \\
\text{target\_fin\_bond} &= w^*_b \cdot S_t - HC^{\text{bond}} + L^{\text{bond}} \\
\text{target\_fin\_cash} &= w^*_c \cdot S_t - HC^{\text{cash}} + L^{\text{cash}}
\end{align}

\textbf{Derivation:} The right-hand side represents the target dollar position in each asset class:
\begin{itemize}
    \item $w^*_i \cdot S_t$: target allocation of surplus to asset class $i$
    \item $-HC^{\text{stock/bond/cash}}$: subtract implicit exposure from human capital
    \item $+L^{\text{bond/cash}}$: add positions needed to hedge the expense liability
\end{itemize}

\textbf{When $NW_t < 0$} (net worth is negative): surplus $S_t = 0$, so the stock target is $-HC^{\text{stock}} \leq 0$. After clipping negatives to zero (Section~\ref{sec:weight_truncation}), the financial portfolio holds zero stocks and allocates entirely to bonds and cash to hedge the expense liability. This is the conservative ``underwater'' behavior: the agent cannot take equity risk until net worth recovers.

\subsection{Weight Truncation (Leverage Constraint)}
\label{sec:weight_truncation}

The target financial holdings can be negative (e.g., when human capital already provides more stock exposure than the optimum requires). We impose a \textbf{leverage cap} governed by the parameter $\lambda_{\max}$ (default 1.0):

\textbf{Step 1: Clip negative risky-asset targets to zero} (no short selling of stocks or bonds):
\begin{align}
\hat{d}_{\text{stock}} &= \max(0,\; \text{target\_fin\_stock}) \\
\hat{d}_{\text{bond}} &= \max(0,\; \text{target\_fin\_bond})
\end{align}

\textbf{Step 2: Cap total long exposure}
\begin{equation}
\text{If } \hat{d}_{\text{stock}} + \hat{d}_{\text{bond}} > \lambda_{\max} \cdot FW_t, \quad \text{scale both down proportionally}
\end{equation}

\textbf{Step 3: Cash is the residual}
\begin{equation}
\hat{d}_{\text{cash}} = FW_t - \hat{d}_{\text{stock}} - \hat{d}_{\text{bond}}
\end{equation}

Cash can be negative when $\lambda_{\max} > 1$, representing borrowing at the short rate.

\textbf{Portfolio weights:}
\begin{equation}
w^{\text{final}}_i = \hat{d}_i / FW_t
\end{equation}

\textbf{Key values of $\lambda_{\max}$:}
\begin{itemize}
    \item $\lambda_{\max} = 1$: No borrowing. Cash is always non-negative. This is the default.
    \item $\lambda_{\max} = 2$: Can borrow up to $1\times FW$ (total risky exposure up to $2 \times FW$).
    \item $\lambda_{\max} = \infty$: Unconstrained (no leverage limit).
\end{itemize}

%----------------------------------------------------------------------
\section{Consumption Rule}
%----------------------------------------------------------------------

\subsection{Consumption Function}

Consumption depends on net worth:
\begin{equation}
C_t = \underbrace{E_t}_{\text{subsistence}} + \underbrace{c_t \cdot \max(0, NW_t)}_{\text{variable}}
\end{equation}

The consumption rate $c_t$ is computed dynamically at each time step from the current portfolio and market state.

\textbf{Expected portfolio return} (arithmetic mean, conditional on time-$t$ information):
\begin{equation}
\mathbb{E}_t[R^p_{t+1}] = w_s(r_t + \mu_s) + w_b(r_t + \mu_b) + w_c \cdot r_t
\end{equation}

\textbf{Portfolio variance} (using realized post-constraint weights $w_s, w_b, w_c$):
\begin{equation}
\text{Var}_t(R^p_{t+1}) = w_s^2 \sigma_s^2 + w_b^2 (D \, \sigma_r)^2 + 2 w_s w_b (-D \, \sigma_s \, \sigma_r \, \rho)
\end{equation}

\textbf{Consumption rate} (certainty-equivalent return):
\begin{equation}
\label{eq:consumption_rate}
c_t = \mathbb{E}_t[R^p_{t+1}] - \tfrac{1}{2}\text{Var}_t(R^p_{t+1}) + \delta
\end{equation}
where $\delta$ is an optional consumption boost parameter (default 0).

\textbf{Key features:}
\begin{itemize}
    \item \textbf{Jensen's correction} ($-\frac{1}{2}\text{Var}_t$): converts the arithmetic mean return to the median (geometric) return, which governs long-run wealth growth. Without this correction, the consumption rate would be too high and wealth would deplete on average.
    \item \textbf{Dynamic rate}: uses the current short rate $r_t$ (not the long-run mean $\bar{r}$), so consumption responds to the prevailing interest rate environment.
    \item \textbf{Realized weights}: uses the post-constraint portfolio weights $w_s, w_b, w_c$, not the unconstrained MV targets. This ensures the variance correction matches the actual portfolio risk.
\end{itemize}

\subsection{Annuity-Adjusted Consumption (Optional)}

When the \texttt{annuity\_consumption} flag is enabled (``Make the Last Check Bounce''), the consumption rate is replaced by the finite-horizon Merton solution:
\begin{equation}
c_t = \frac{1}{A(T - t)}, \quad \text{where } A(T-t) = \frac{1 - (1 + r_{CE})^{-(T-t)}}{r_{CE}}
\end{equation}
and $r_{CE}$ is the certainty-equivalent return from Eq.~\ref{eq:consumption_rate}. The annuity factor $A(T-t)$ decreases as end-of-life approaches, so the consumption rate \emph{increases} with age, converging to ``consume everything'' in the final year. This produces a lifecycle plan that spends wealth down to approximately zero at $T$ (end\_age).

\textbf{Default behavior.} With \texttt{annuity\_consumption = False} (default), the consumption rate is simply $c_t = r_{CE}$, which is the infinite-horizon case. The infinite-horizon rate is \emph{horizon-independent}: a 25-year-old and a 90-year-old with the same portfolio and rates consume at the same rate of net worth.

\subsection{Constraints}

\textbf{Working years ($t < T_R$):}
\begin{itemize}
    \item Cannot consume more than earnings: $C_t \leq Y_t$
\end{itemize}

\textbf{Retirement ($t \geq T_R$):}
\begin{itemize}
    \item Cannot consume more than financial wealth: $C_t \leq FW_t$
    \item If $FW_t \leq 0$: default (set $C_t = 0$ for all future periods)
\end{itemize}

\subsection{Wealth Evolution}

\begin{equation}
FW_{t+1} = FW_t \cdot (1 + R^p_{t+1}) + S_t
\end{equation}
where:
\begin{itemize}
    \item $R^p_{t+1} = w_s R^s_{t+1} + w_b R^b_{t+1} + w_c \, r_t$ (portfolio return using realized asset returns)
    \item $S_t = Y_t - C_t$ (savings; negative in retirement when consuming from wealth)
\end{itemize}

%----------------------------------------------------------------------
\section{Simulation Architecture}
%----------------------------------------------------------------------

\subsection{Unified Engine}

The model uses a single simulation engine for both deterministic and stochastic analysis:
\begin{itemize}
    \item \textbf{Zero-shock path} (called ``median path'' in the code): Pass $\varepsilon^r = \varepsilon^s = 0$ for all $t$.
    \item \textbf{Monte Carlo}: Pass random shocks drawn from $N(0,1)$ with correlation $\rho$. Produces stochastic paths with dynamic revaluation of HC and $L$ at each step.
\end{itemize}

\subsection{Jensen's Correction: Consistent Geometric Returns}
\label{sec:jensen}

Both the consumption rate and wealth evolution use the \textbf{geometric (median) return}:
\[
R^p_{\text{geom}} = \mathbb{E}_t[R^p_{t+1}] - \tfrac{1}{2}\text{Var}_t(R^p_{t+1})
\]
where
\[
\mathbb{E}_t[R^p_{t+1}] = w_s(r + \mu_s) + w_b(r + \mu_b) + w_c \cdot r, \qquad
\text{Var}_t(R^p_{t+1}) = w_s^2\sigma_s^2 + w_b^2(D\sigma_r)^2 + 2w_sw_b\text{Cov}(R^s,R^b)
\]
The Jensen's correction $\tfrac{1}{2}\text{Var}(R^p)$ varies at each time step because portfolio weights change over the lifecycle (stock-heavy when young due to HC, bond/cash-heavy when old).

\textbf{Zero-shock path} (``median path''). With $\varepsilon^r = \varepsilon^s = 0$:
\begin{itemize}
    \item The consumption rate is $c_t = R^p_{\text{geom}} + \delta$ (Eq.~\ref{eq:consumption_rate})
    \item Wealth evolves at the same geometric return: $FW_{t+1} = FW_t(1 + R^p_{\text{geom}}) + Y_t - C_t$
    \item This produces a \textbf{true median trajectory}: the path that the median Monte Carlo simulation converges to as $N \to \infty$
\end{itemize}

\textbf{Monte Carlo.} With random shocks:
\begin{itemize}
    \item Each asset earns a realized return that includes shocks
    \item Wealth evolves at the realized (stochastic) portfolio return
    \item The median across paths $\approx$ the zero-shock geometric path
\end{itemize}

The correction is largest early in life when the portfolio is stock-heavy. With default parameters and ${\sim}100\%$ stocks: correction $= \tfrac{1}{2}(0.18)^2 = 1.62$ pp/yr. At mid-career (${\sim}69\%$ stocks): correction $\approx 0.79$ pp/yr.

\subsection{Time Step Sequence}

At each period $t$, the simulation proceeds in order:
\begin{enumerate}
    \item Observe current state: $FW_t$, $r_t$, wage multiplier $W_t$
    \item Compute $HC_t$, $L_t$, $NW_t$ using current rate $r_t$
    \item Decompose HC and $L$ into stock/bond/cash components
    \item Compute target financial holdings via surplus optimization
    \item Normalize to portfolio weights $(w_s, w_b, w_c)$ with leverage constraint
    \item Compute dynamic consumption rate $c_t$ using realized weights and Jensen's correction
    \item Compute consumption $C_t$ and apply constraints
    \item Evolve wealth: $FW_{t+1} = FW_t(1 + R^p_{t+1}) + Y_t - C_t$. For zero-shock paths, $R^p_{t+1} = R^p_{\text{geom}}$; for MC, $R^p_{t+1}$ is the realized stochastic return
\end{enumerate}

The order is important: portfolio weights are determined \emph{before} consumption, so the variance correction in the consumption rate uses the actual portfolio allocation.

%----------------------------------------------------------------------
\section{Parameter Summary}
%----------------------------------------------------------------------

\begin{table}[h]
\centering
\caption{Economic Parameters (\texttt{EconomicParams})}
\begin{tabular}{llll}
\toprule
\textbf{Parameter} & \textbf{Symbol} & \textbf{Default} & \textbf{Description} \\
\midrule
\texttt{r\_bar} & $\bar{r}$ & 0.02 & Long-run mean interest rate (= initial rate $r_0$) \\
\texttt{phi} & $\phi$ & 1.0 & Rate persistence (1.0 = random walk) \\
\texttt{sigma\_r} & $\sigma_r$ & 0.003 & Interest rate volatility (0.3 pp) \\
\texttt{mu\_excess} & $\mu_s$ & 0.045 & Equity risk premium (4.5 pp) \\
\texttt{sigma\_s} & $\sigma_s$ & 0.18 & Stock return volatility \\
\texttt{bond\_sharpe} & $\text{SR}_b$ & 0.0 & Bond Sharpe ratio ($\mu_b = \text{SR}_b \cdot D \cdot \sigma_r$) \\
\texttt{bond\_duration} & $D$ & 20.0 & Duration of bond portfolio \\
\texttt{rho} & $\rho$ & 0.0 & Correlation between rate and stock shocks \\
\texttt{max\_duration} & $D_{\max}$ & None & Optional cap on computed durations \\
\bottomrule
\end{tabular}
\end{table}

\begin{table}[h]
\centering
\caption{Lifecycle Parameters (\texttt{LifecycleParams})}
\begin{tabular}{llll}
\toprule
\textbf{Parameter} & \textbf{Symbol} & \textbf{Default} & \textbf{Description} \\
\midrule
\texttt{start\_age} & $t_0$ & 25 & Career start age \\
\texttt{retirement\_age} & $T_R$ & 65 & Retirement age \\
\texttt{end\_age} & $T$ & 95 & Planning horizon / end of life \\
\texttt{initial\_earnings} & $Y_0$ & 200 & Initial annual earnings (\$K) \\
\texttt{base\_expenses} & $E_0$ & 100 & Base annual expenses (\$K) \\
\texttt{retirement\_expenses} & $E_R$ & 100 & Retirement subsistence expenses (\$K) \\
\texttt{initial\_wealth} & $FW_0$ & 100 & Initial financial wealth (\$K) \\
\texttt{gamma} & $\gamma$ & 2.0 & Risk aversion coefficient \\
\texttt{stock\_beta\_human\_capital} & $\beta_{HC}$ & 0.0 & Stock beta of human capital \\
\texttt{max\_leverage} & $\lambda_{\max}$ & 1.0 & Max risky-asset exposure / $FW$ \\
\texttt{consumption\_boost} & $\delta$ & 0.0 & Consumption rate boost \\
\texttt{annuity\_consumption} & --- & False & Annuity-adjust consumption to spend down by $T$ \\
\bottomrule
\end{tabular}
\end{table}

\textbf{Notes:}
\begin{itemize}
    \item The consumption rate $c_t$ is derived dynamically (Eq.~\ref{eq:consumption_rate}), not set as a parameter.
    \item With default parameters ($g_Y = d_Y = g_E = 0$), earnings are flat at \$200K and expenses are flat at \$100K.
    \item The bond risk premium is derived: $\mu_b = \text{SR}_b \cdot D \cdot \sigma_r$. With default $\text{SR}_b = 0$, bonds earn no excess return.
\end{itemize}

\end{document}
