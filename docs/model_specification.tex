\documentclass[11pt]{article}
\usepackage[margin=1in]{geometry}
\usepackage{amsmath, amssymb}
\usepackage{booktabs}
\usepackage{hyperref}

\title{Lifecycle Investment Model Specification\\
\large Random Walk Interest Rate Case ($\phi = 1$)}
\author{FINC450}
\date{}

\begin{document}

\maketitle

\section{Data Generating Process}

This document describes the lifecycle investment model with \textbf{random walk interest rates} ($\phi = 1$). Under this specification, rates follow a pure random walk with no mean reversion.

\subsection{Interest Rate Process (Random Walk)}

The short-term interest rate evolves as:
\begin{equation}
r_{t+1} = r_t + \sigma_r \varepsilon^r_{t+1}, \quad \varepsilon^r_{t+1} \sim N(0,1)
\end{equation}
where:
\begin{itemize}
    \item $\sigma_r$ = interest rate volatility
    \item No drift term (expected change is zero)
    \item No mean reversion ($\phi = 1$ implies persistence = 1)
    \item Floor at $r_{\text{floor}}$ to prevent negative rates
\end{itemize}

\textbf{Key implication:} With $\phi = 1$, the effective Macaulay duration equals maturity, and bond pricing simplifies to standard discounting. Modified duration is then maturity divided by $(1+r)$.

\subsection{Stock Returns}

Stock returns from $t$ to $t+1$:
\begin{equation}
R^s_{t+1} = r_t + \mu_s + \sigma_s \varepsilon^s_{t+1}
\end{equation}
where:
\begin{itemize}
    \item $\mu_s$ = equity risk premium (excess return over short rate)
    \item $\sigma_s$ = stock volatility
    \item $\text{Corr}(\varepsilon^r_{t+1}, \varepsilon^s_{t+1}) = \rho$ (typically small or zero)
\end{itemize}

\subsection{Bond Returns (Modified Duration Approximation)}

For a bond with modified duration $D^{\text{mod}}$, the return from $t$ to $t+1$ is:
\begin{equation}
R^b_{t+1} \approx r_t + \mu_b - D^{\text{mod}} \cdot \Delta r_{t+1}
\end{equation}
where:
\begin{itemize}
    \item $D^{\text{mod}}$ = modified duration (not Macaulay duration)
    \item $\mu_b$ = bond risk premium (term spread over short rate)
    \item $\Delta r_{t+1} = r_{t+1} - r_t = \sigma_r \varepsilon^r_{t+1}$
\end{itemize}

Substituting the rate change:
\begin{equation}
R^b_{t+1} \approx r_t + \mu_b - D^{\text{mod}} \cdot \sigma_r \varepsilon^r_{t+1}
\end{equation}

\textbf{Note:} Modified duration $D^{\text{mod}} = D^{\text{Mac}}/(1+r)$ gives the percentage price change per unit change in yield. Using modified duration directly in the return equation is correct; using Macaulay duration would overstate interest rate sensitivity.

\textbf{Intuition:} When rates rise ($\varepsilon^r_{t+1} > 0$), bond prices fall, generating a negative capital gain proportional to modified duration.

\textbf{Bond risk premium.} The bond excess return is parameterized via the Bond Sharpe ratio $\text{SR}_b$:
\begin{equation}
\mu_b = \text{SR}_b \cdot D^{\text{mod}} \cdot \sigma_r
\end{equation}
The bond return volatility is $\sigma_b = D^{\text{mod}} \cdot \sigma_r$, so $\mu_b = \text{SR}_b \cdot \sigma_b$. The default is $\text{SR}_b = 0$ (expectations hypothesis: no term premium).

%----------------------------------------------------------------------
\section{Portfolio Allocation Rule}
%----------------------------------------------------------------------

\subsection{Human Capital}

Human Capital (HC) is the present value of future labor earnings, discounted at the current rate:
\begin{equation}
HC_t = \sum_{s=t}^{T_R-1} \frac{Y_s}{(1+r_t)^{s-t}}
\end{equation}
where:
\begin{itemize}
    \item $Y_s$ = earnings at age $s$
    \item $T_R$ = retirement age
    \item $r_t$ = current interest rate
\end{itemize}

After retirement ($t \geq T_R$): $HC_t = 0$.

\subsection{Present Value of Expenses (Liability)}

The present value of future subsistence expenses:
\begin{equation}
L_t = PV_t(\text{expenses}) = \sum_{s=t}^{T-1} \frac{E_s}{(1+r_t)^{s-t}}
\end{equation}
where:
\begin{itemize}
    \item $E_s$ = subsistence expenses at age $s$
    \item $T$ = end age (death)
    \item $r_t$ = current interest rate
\end{itemize}

\subsection{Total Wealth (Net Worth)}

Total wealth is defined as assets minus liabilities:
\begin{equation}
TW_t = FW_t + HC_t - L_t
\end{equation}
where:
\begin{itemize}
    \item $FW_t$ = Financial Wealth (liquid portfolio)
    \item $HC_t$ = Human Capital (implicit bond-like asset)
    \item $L_t$ = Present value of future expenses (liability)
\end{itemize}

This is equivalent to \textbf{net worth}---the amount available for discretionary consumption beyond subsistence.

\subsection{Target Total Allocation (Mean-Variance Optimization)}

The optimal portfolio weights from mean-variance optimization:
\begin{equation}
\mathbf{w}^* = \frac{1}{\gamma} \Sigma^{-1} \boldsymbol{\mu}
\end{equation}

\textbf{Explicit 2-asset system.} Cash earns the risk-free rate, so we solve for stock and bond weights over cash. The excess return vector is:
\begin{equation}
\boldsymbol{\mu} = \begin{pmatrix} \mu_s \\ \mu_b \end{pmatrix}
\end{equation}

The covariance matrix of risky asset returns uses $\sigma_b = D^{\text{mod}} \cdot \sigma_r$ for bond volatility:
\begin{equation}
\Sigma = \begin{pmatrix}
\sigma_s^2 & -D^{\text{mod}} \sigma_s \sigma_r \rho \\
-D^{\text{mod}} \sigma_s \sigma_r \rho & (D^{\text{mod}} \sigma_r)^2
\end{pmatrix}
\end{equation}

The off-diagonal term reflects $\text{Cov}(R^s_{t+1}, R^b_{t+1}) = -D^{\text{mod}} \sigma_s \sigma_r \rho$: the negative sign arises because rising rates ($\varepsilon^r_{t+1} > 0$) hurt bond returns.

Solving $\mathbf{w}^* = \frac{1}{\gamma} \Sigma^{-1} \boldsymbol{\mu}$ gives the optimal stock and bond weights $(w^*_s, w^*_b)$, with cash as the residual:
\begin{equation}
w^*_c = 1 - w^*_s - w^*_b
\end{equation}

These MV weights are used \emph{unconstrained} as target allocations. Constraints are applied only to the final dollar-level portfolio holdings (see Section~\ref{sec:weight_truncation}).

\subsection{Modified Duration of Human Capital and Expenses}

HC and expenses have interest rate sensitivity (duration). We use \textbf{modified duration} throughout for consistency with the bond return approximation.

\textbf{Modified Duration of Human Capital:}
\begin{equation}
D^{HC,\text{mod}}_t = \frac{1}{1+r_t} \cdot \frac{\sum_{s=t}^{T_R-1} (s-t) \cdot \frac{Y_s}{(1+r_t)^{s-t}}}{HC_t}
\end{equation}

\textbf{Modified Duration of Expenses:}
\begin{equation}
D^{L,\text{mod}}_t = \frac{1}{1+r_t} \cdot \frac{\sum_{s=t}^{T-1} (s-t) \cdot \frac{E_s}{(1+r_t)^{s-t}}}{L_t}
\end{equation}

\textbf{Note:} The summation term is the Macaulay duration (weighted average time to cash flows). Dividing by $(1+r_t)$ converts to modified duration, which measures the percentage change in present value for a unit change in yield.

\subsection{Human Capital Decomposition}

Assuming HC is $\beta_{HC}$ stock-like (correlated with market) and the rest is bond-like:
\begin{align}
HC^{\text{stock}} &= \beta_{HC} \cdot HC_t \\
HC^{\text{bond}} &= (1-\beta_{HC}) \cdot HC_t \cdot \frac{D^{HC,\text{mod}}_t}{D^{\text{mod}}} \\
HC^{\text{cash}} &= (1-\beta_{HC}) \cdot HC_t \cdot \left(1 - \frac{D^{HC,\text{mod}}_t}{D^{\text{mod}}}\right)
\end{align}
where $D^{\text{mod}}$ = modified duration of bonds in the financial portfolio.

\subsection{Expense Liability Decomposition}

The expense liability is decomposed by modified duration:
\begin{align}
L^{\text{bond}} &= L_t \cdot \frac{D^{L,\text{mod}}_t}{D^{\text{mod}}} \\
L^{\text{cash}} &= L_t \cdot \left(1 - \frac{D^{L,\text{mod}}_t}{D^{\text{mod}}}\right)
\end{align}

\subsection{Target Financial Holdings (Surplus Optimization)}

We define \textbf{surplus} as the positive part of net worth:
\begin{equation}
S_t = \max(0,\; TW_t) = \max(0,\; FW_t + HC_t - L_t)
\end{equation}

The target allocation is applied to surplus rather than raw net worth. This prevents the optimizer from taking leveraged bets when the agent is ``underwater'' ($TW < 0$).

\begin{align}
\text{target\_fin\_stock} &= w^*_{\text{stock}} \cdot S_t - HC^{\text{stock}} \\
\text{target\_fin\_bond} &= w^*_{\text{bond}} \cdot S_t - HC^{\text{bond}} + L^{\text{bond}} \\
\text{target\_fin\_cash} &= w^*_{\text{cash}} \cdot S_t - HC^{\text{cash}} + L^{\text{cash}}
\end{align}

\textbf{Derivation:} The right-hand side represents the target dollar position in each asset class:
\begin{itemize}
    \item $w^*_i \cdot S_t$: target allocation of surplus to asset class $i$
    \item $-HC^{\text{stock/bond/cash}}$: subtract implicit exposure from human capital
    \item $+L^{\text{bond/cash}}$: add positions needed to hedge the expense liability
\end{itemize}

\textbf{When $TW_t < 0$} (net worth is negative): surplus $S_t = 0$, so the stock target is $-HC^{\text{stock}} \leq 0$. After clipping negatives to zero (Section~\ref{sec:weight_truncation}), the financial portfolio holds zero stocks and allocates entirely to bonds and cash to hedge the expense liability. This is the conservative ``underwater'' behavior: the agent cannot take equity risk until net worth recovers.

\subsection{Weight Truncation (No Leverage Constraint)}
\label{sec:weight_truncation}

The target financial holdings can be negative (e.g., when human capital already provides more stock exposure than the optimum requires). We impose a \textbf{no-leverage, no-short-selling constraint}:

\textbf{Step 1: Clip negative dollar targets to zero}
\begin{equation}
\hat{d}_i = \max(0,\; \text{target\_fin}_i), \quad i \in \{\text{stock}, \text{bond}, \text{cash}\}
\end{equation}

\textbf{Step 2: Normalize to portfolio weights}
\begin{equation}
w^{\text{final}}_i = \frac{\hat{d}_i}{\sum_j \hat{d}_j}
\end{equation}

There is no separate $\min(1, \cdot)$ step---the upper bound is enforced automatically by normalization. The clipping operates on \emph{dollar amounts} (not weights), preserving the relative sizing of positive positions.

\textbf{Implications:}
\begin{itemize}
    \item Early career: Stock weight is capped at 100\%, forgoing the theoretically optimal leveraged position
    \item The constraint binds most when $HC/FW$ is high (young workers with little savings)
    \item As financial wealth grows and human capital depletes, the constraint relaxes
    \item This is a practical constraint reflecting real-world borrowing limitations
\end{itemize}

%----------------------------------------------------------------------
\section{Consumption Rule}
%----------------------------------------------------------------------

\subsection{Consumption Function}

Consumption depends on net worth:
\begin{equation}
C_t = \underbrace{E_t}_{\text{subsistence}} + \underbrace{c_t \cdot \max(0, TW_t)}_{\text{variable}}
\end{equation}

The consumption rate $c_t$ is computed dynamically at each time step from the current portfolio and market state.

\textbf{Expected portfolio return} (arithmetic mean, conditional on time-$t$ information):
\begin{equation}
\mathbb{E}_t[R^p_{t+1}] = w_s(r_t + \mu_s) + w_b(r_t + \mu_b) + w_c \cdot r_t
\end{equation}

\textbf{Portfolio variance} (using realized post-constraint weights $w_s, w_b, w_c$):
\begin{equation}
\text{Var}_t(R^p_{t+1}) = w_s^2 \sigma_s^2 + w_b^2 (D^{\text{mod}} \sigma_r)^2 + 2 w_s w_b (-D^{\text{mod}} \sigma_s \sigma_r \rho)
\end{equation}

\textbf{Consumption rate} (certainty-equivalent return):
\begin{equation}
\label{eq:consumption_rate}
c_t = \mathbb{E}_t[R^p_{t+1}] - \tfrac{1}{2}\text{Var}_t(R^p_{t+1}) + \delta
\end{equation}
where $\delta$ is an optional consumption boost parameter (default 0).

\textbf{Key features:}
\begin{itemize}
    \item \textbf{Jensen's correction} ($-\frac{1}{2}\text{Var}_t$): converts the arithmetic mean return to the median (geometric) return, which governs long-run wealth growth. Without this correction, the consumption rate would be too high and wealth would deplete on average.
    \item \textbf{Dynamic rate}: uses the current short rate $r_t$ (not the long-run mean $\bar{r}$), so consumption responds to the prevailing interest rate environment.
    \item \textbf{Realized weights}: uses the post-constraint portfolio weights $w_s, w_b, w_c$, not the unconstrained MV targets. This ensures the variance correction matches the actual portfolio risk.
\end{itemize}

\subsection{Constraints}

\textbf{Working years ($t < T_R$):}
\begin{itemize}
    \item Cannot consume more than earnings: $C_t \leq Y_t$
\end{itemize}

\textbf{Retirement ($t \geq T_R$):}
\begin{itemize}
    \item Cannot consume more than financial wealth: $C_t \leq FW_t$
    \item If $FW_t \leq 0$: default (set $C_t = 0$)
\end{itemize}

\subsection{Wealth Evolution}

\begin{equation}
FW_{t+1} = FW_t \cdot (1 + R^p_{t+1}) + S_t
\end{equation}
where:
\begin{itemize}
    \item $R^p_{t+1} = w_s R^s_{t+1} + w_b R^b_{t+1} + w_c \, r_t$ (portfolio return)
    \item $S_t = Y_t - C_t$ (savings; can be negative in retirement)
\end{itemize}

%----------------------------------------------------------------------
\section{Parameter Summary}
%----------------------------------------------------------------------

\begin{table}[h]
\centering
\begin{tabular}{lll}
\toprule
\textbf{Parameter} & \textbf{Symbol} & \textbf{Description} \\
\midrule
\multicolumn{3}{l}{\textit{Economic Parameters}} \\
$r_0$ & & Initial interest rate \\
$\sigma_r$ & & Interest rate volatility \\
$\phi$ & & Rate persistence (= 1 for random walk) \\
$\mu_s$ & & Equity risk premium \\
$\sigma_s$ & & Stock volatility \\
$\text{SR}_b$ & & Bond Sharpe ratio (default 0; $\mu_b = \text{SR}_b \cdot D^{\text{mod}} \cdot \sigma_r$) \\
$D^{\text{mod}}$ & & Bond modified duration \\
$\rho$ & & Stock-rate correlation \\
\midrule
\multicolumn{3}{l}{\textit{Lifecycle Parameters}} \\
$T_R$ & & Retirement age \\
$T$ & & End age \\
$Y_t$ & & Earnings profile \\
$E_t$ & & Expense profile \\
$\beta_{HC}$ & & Stock beta of human capital \\
$\gamma$ & & Risk aversion \\
$\delta$ & & Consumption boost (default 0) \\
\multicolumn{3}{l}{\quad\textit{Note: consumption rate $c_t$ is derived dynamically (Eq.~\ref{eq:consumption_rate})}} \\
\bottomrule
\end{tabular}
\end{table}

\end{document}
