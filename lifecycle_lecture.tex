\documentclass[aspectratio=169,11pt]{beamer}

% Theme and colors
\usetheme{Madrid}
\usecolortheme{seahorse}
\setbeamertemplate{navigation symbols}{}
\setbeamertemplate{footline}[frame number]

% Packages
\usepackage{graphicx}
\usepackage{amsmath}
\usepackage{booktabs}
\usepackage{tikz}

% Custom colors
\definecolor{darkblue}{RGB}{52, 73, 94}
\definecolor{accent}{RGB}{41, 128, 185}
\setbeamercolor{title}{fg=darkblue}
\setbeamercolor{frametitle}{fg=darkblue}

% Title information
\title[Lifecycle Investing]{Lifecycle Investing: \\[0.3em] A Finance Theory Perspective}
\subtitle{FINC 450}
\author{}
\date{}

\begin{document}

% =============================================================================
% TITLE
% =============================================================================
\begin{frame}
\titlepage
\end{frame}

% =============================================================================
% OUTLINE
% =============================================================================
\begin{frame}{Today's Lecture}
\tableofcontents
\end{frame}

% =============================================================================
% SECTION 1: THE PROBLEM
% =============================================================================
\section{The Lifecycle Problem}

\begin{frame}{The Fundamental Problem}
\begin{center}
\Large
\textbf{Income and expenses don't match over time.}
\end{center}

\vspace{1em}

\begin{itemize}
    \item You earn during working years (ages 25--65)
    \item You consume throughout your entire life (ages 25--85+)
    \item This \textbf{mismatch} is THE problem lifecycle finance solves
\end{itemize}
\end{frame}

\begin{frame}{Income vs. Expenses Over the Lifecycle}
\begin{center}
\includegraphics[width=0.85\textwidth]{figures/01_income_expenses.png}
\end{center}
\end{frame}

\begin{frame}{The Cash Flow Pattern}
\begin{center}
\includegraphics[width=0.85\textwidth]{figures/02_cash_flow.png}
\end{center}

\vspace{0.5em}
\small
\textbf{Key insight:} The pattern of cash flows determines the investment problem.
\end{frame}

% =============================================================================
% SECTION 2: THE HIDDEN BALANCE SHEET
% =============================================================================
\section{Your Hidden Balance Sheet}

\begin{frame}{Thinking in Present Values}
\begin{center}
\Large
\textbf{Finance thinks in present values.}
\end{center}

\vspace{1em}

\begin{columns}
\begin{column}{0.5\textwidth}
\textbf{Assets:}
\begin{itemize}
    \item Financial wealth (savings)
    \item \textcolor{accent}{\textbf{Human capital}} (PV of future earnings)
\end{itemize}
\end{column}
\begin{column}{0.5\textwidth}
\textbf{Liabilities:}
\begin{itemize}
    \item \textcolor{red}{\textbf{Expense liability}} (PV of future spending needs)
\end{itemize}
\end{column}
\end{columns}

\vspace{1em}
\begin{center}
\fbox{\textbf{Net Worth} = Human Capital + Financial Wealth $-$ PV(Expenses)}
\end{center}
\end{frame}

\begin{frame}{Present Values: Your Hidden Balance Sheet}
\begin{center}
\includegraphics[width=0.85\textwidth]{figures/03_present_values.png}
\end{center}
\end{frame}

\begin{frame}{Human Capital is Your Biggest Asset (Early On)}
\begin{center}
\includegraphics[width=0.85\textwidth]{figures/04_wealth_composition.png}
\end{center}

\vspace{0.5em}
\small
\textbf{Key insight:} At age 25, most of your wealth is human capital---you just can't see it in your brokerage account.
\end{frame}

% =============================================================================
% SECTION 3: THE GAUGES FRAMEWORK
% =============================================================================
\section{The Four Gauges}

\begin{frame}{Beyond the 401(k) Balance}
\begin{center}
\Large
\textbf{It's not just about your retirement account.}
\end{center}

\vspace{1em}

Traditional advice focuses on one number: your savings balance.

\vspace{1em}

Finance theory says you need to track \textbf{four gauges}:
\begin{enumerate}
    \item Human Capital (your future earning power)
    \item Financial Wealth (your savings)
    \item Expense Liability (what you owe your future self)
    \item Net Worth (assets minus liabilities)
\end{enumerate}
\end{frame}

\begin{frame}{The Four Gauges of Lifecycle Finance}
\begin{center}
\includegraphics[width=0.95\textwidth]{figures/05_gauges_dashboard.png}
\end{center}
\end{frame}

\begin{frame}{Why Track All Four?}
\begin{columns}
\begin{column}{0.5\textwidth}
\textbf{Human Capital (Gauge 1):}
\begin{itemize}
    \item Depletes as you age
    \item Affected by career risk
    \item Has duration (interest rate sensitivity)
\end{itemize}

\vspace{1em}

\textbf{Financial Wealth (Gauge 2):}
\begin{itemize}
    \item What you control directly
    \item Grows through savings + returns
    \item Must replace HC over time
\end{itemize}
\end{column}
\begin{column}{0.5\textwidth}
\textbf{Expense Liability (Gauge 3):}
\begin{itemize}
    \item Your commitment to future self
    \item Has duration too!
    \item Determines ``fully funded'' status
\end{itemize}

\vspace{1em}

\textbf{Net Worth (Gauge 4):}
\begin{itemize}
    \item The bottom line
    \item Drives consumption decisions
    \item Should stay positive!
\end{itemize}
\end{column}
\end{columns}
\end{frame}

% =============================================================================
% SECTION 4: WHY ALLOCATION CHANGES
% =============================================================================
\section{Why Portfolio Allocation Changes Over Life}

\begin{frame}{The Key Insight}
\begin{center}
\Large
\textbf{Human capital is like a bond.}
\end{center}

\vspace{1em}

\begin{itemize}
    \item Stable, predictable income stream (for most people)
    \item Has \textbf{duration}: sensitive to interest rates
    \item Beta to stock market $\approx 0$ for professors, consultants
    \item Beta $> 0$ for entrepreneurs, tech workers
\end{itemize}

\vspace{1em}

\begin{center}
\fbox{
\parbox{0.8\textwidth}{
\centering
\textbf{Implication:} To maintain target total risk, \\
financial portfolio must adjust as HC depletes.
}
}
\end{center}
\end{frame}

\begin{frame}{Why Portfolio Allocation Changes Over Life}
\begin{center}
\includegraphics[width=0.95\textwidth]{figures/06_allocation_theory.png}
\end{center}
\end{frame}

\begin{frame}{The Logic of the Glide Path}

\textbf{Young investor (age 25):}
\begin{itemize}
    \item Total wealth = 90\% Human Capital + 10\% Financial Wealth
    \item HC is bond-like $\Rightarrow$ already have implicit bond position
    \item Financial portfolio should be \textbf{100\% stocks} to balance
\end{itemize}

\vspace{1em}

\textbf{Retiree (age 70):}
\begin{itemize}
    \item Total wealth = 0\% Human Capital + 100\% Financial Wealth
    \item No implicit bond position from HC
    \item Financial portfolio should match \textbf{target allocation} (e.g., 50/50)
\end{itemize}

\vspace{1em}

\begin{center}
\textcolor{accent}{\textbf{This is WHY target-date funds have a ``glide path''!}}
\end{center}
\end{frame}

% =============================================================================
% SECTION 5: LABOR INCOME RISK
% =============================================================================
\section{Labor Income Risk Matters}

\begin{frame}{Not All Human Capital is the Same}
\begin{center}
\Large
\textbf{Your job's riskiness matters.}
\end{center}

\vspace{1em}

\begin{columns}
\begin{column}{0.5\textwidth}
\textbf{Low Beta ($\beta \approx 0$):}
\begin{itemize}
    \item Tenured professor
    \item Government employee
    \item Doctor, lawyer
\end{itemize}
HC is bond-like \\
$\Rightarrow$ Hold \textbf{more} stocks
\end{column}
\begin{column}{0.5\textwidth}
\textbf{High Beta ($\beta \approx 1$):}
\begin{itemize}
    \item Tech startup founder
    \item Investment banker
    \item Sales (commission-based)
\end{itemize}
HC is stock-like \\
$\Rightarrow$ Hold \textbf{fewer} stocks
\end{column}
\end{columns}
\end{frame}

\begin{frame}{Effect of Labor Income Risk on Portfolio Choice}
\begin{center}
\includegraphics[width=0.95\textwidth]{figures/09_beta_comparison.png}
\end{center}
\end{frame}

\begin{frame}{Practical Implications}

\textbf{If your income is stable (professor, government):}
\begin{itemize}
    \item You can afford more risk in your portfolio
    \item 100\% stocks early in career is reasonable
    \item Your human capital provides diversification
\end{itemize}

\vspace{1em}

\textbf{If your income is risky (entrepreneur, tech):}
\begin{itemize}
    \item Be more conservative in your portfolio
    \item Don't double down on market risk
    \item Consider your company stock exposure carefully!
\end{itemize}
\end{frame}

% =============================================================================
% SECTION 6: DURATION AND INTEREST RATES
% =============================================================================
\section{Duration Matching}

\begin{frame}{Interest Rate Risk on Your Balance Sheet}
\begin{center}
\includegraphics[width=0.85\textwidth]{figures/07_duration_matching.png}
\end{center}

\vspace{0.5em}
\small
\textbf{Duration} measures interest rate sensitivity. Match asset and liability durations to hedge rate risk.
\end{frame}

% =============================================================================
% SECTION 7: CONSUMPTION
% =============================================================================
\section{Optimal Consumption}

\begin{frame}{Consumption Smoothing}
\begin{center}
\includegraphics[width=0.85\textwidth]{figures/08_consumption_path.png}
\end{center}

\vspace{0.5em}
\small
\textbf{Optimal:} Consume a fraction of net worth. Automatically adjusts to market conditions.
\end{frame}

% =============================================================================
% SUMMARY
% =============================================================================
\section{Summary}

\begin{frame}{Key Takeaways}

\begin{enumerate}
    \item \textbf{The Problem:} Income and expenses don't match over time

    \vspace{0.5em}

    \item \textbf{Human Capital:} Your biggest asset early in life (but invisible)

    \vspace{0.5em}

    \item \textbf{Four Gauges:} Track HC, FW, Expenses, and Net Worth---not just your 401(k)

    \vspace{0.5em}

    \item \textbf{Glide Path Logic:} HC is bond-like $\Rightarrow$ young people should hold more stocks

    \vspace{0.5em}

    \item \textbf{Labor Risk Matters:} Risky job $\Rightarrow$ more conservative portfolio

    \vspace{0.5em}

    \item \textbf{Duration:} Match asset and liability durations to manage interest rate risk
\end{enumerate}
\end{frame}

\begin{frame}{The Merton Framework}

\textbf{Optimal portfolio allocation:}
\[
w^* = \frac{1}{\gamma} \Sigma^{-1} \mu
\]

where:
\begin{itemize}
    \item $\gamma$ = risk aversion coefficient
    \item $\Sigma$ = covariance matrix of asset returns
    \item $\mu$ = vector of expected excess returns
\end{itemize}

\vspace{1em}

\textbf{Key extension for lifecycle:}
\begin{itemize}
    \item Human capital is an implicit asset in your portfolio
    \item Financial portfolio adjusts to reach total wealth target
    \item As HC depletes, financial portfolio converges to $w^*$
\end{itemize}
\end{frame}

\begin{frame}{}
\begin{center}
\Huge
\textbf{Questions?}
\end{center}
\end{frame}

\end{document}
